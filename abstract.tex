\documentclass[12pt]{article}
\usepackage{amsmath}
\usepackage{amssymb}
\usepackage[margin=1in]{geometry}
\usepackage{setspace}
\doublespacing

\author{Tyler Reisinger}
\title{Draft Abstract}
\date{}

\begin{document}
\maketitle

"Hot Jupiters" (HJs) are large planets that orbit very close to their host star. 
They appear to be somewhat commonly occurring in the universe with about 1\% 
of sun-like solar systems containing one (A. Brucalassi et al. 2016). 
However, the mechanism that causes these planets to have such extreme orbital parameters
is not well known. 
We aim to explore the role star-planet and planet-planet interactions 
within a star cluster may have in producing Hot 
Jupiters. By using computational simulations of "typical" star clusters,
we will attempt to probe what happens over vast periods of time
in star clusters similar to those found within the observable universe.
We will run detailed simulations of close stellar encounters within these clusters
and observe the resulting state of the systems after the stars have separated. 
As a result, we hope to observe the formation of Hot Jupiter-like orbits that
match observational data. Additionally, we will observe what other damage might 
persist within the planetary orbits of the system that can be used as 
observational evidence for this formation scenario.
\end{document}
