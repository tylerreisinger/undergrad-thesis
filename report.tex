\documentclass[12pt]{article}
\usepackage{amsmath}
\usepackage[margin=1in]{geometry}
\usepackage{setspace}
\doublespacing

\author{Tyler Reisinger}
\title{Effect of Jupiter-sized Planet Interactions on Planetary Systems}
\date{}

\begin{document}
\maketitle

\section*{Abstract}

"Hot Jupiters" (HJs) are large planets that orbit very close to their host star. 
They appear to be somewhat commonly occurring in the universe with about 1\% 
of sun-like solar systems containing one (A. Brucalassi et al. 2016). However, the mechanism that causes these planets to have such extreme orbital parameters
is not well known. We aim to explore the role planetary interactions may have in producing Hot 
Jupiter systems within a star cluster via computational simulations of these interactions.
In addition, we will observe what sort of effects that the interactions have on the rest of the star system.

\section*{Report}

In the study of exoplanets, one interesting class of planets are the so-called "Hot Jupiters" -- planets with near-Jupiter mass that have a 
very small radius orbit around their host star. Hot Jupiters are interesting as their mode of formation is not fully understood and the formation process likely leaves
a visible long-term impact on the host planetary system. 
Despite their exotic orbit and uncertainties around their formation, 
they appear to be somewhat common within the universe as observations and previous models place the occurrence of Hot Jupiters at around 1\% of all solar systems (A. Brucalassi et al. 2016).  

One current theory about the formation of these Hot Jupiters is that they are not formed in place, but rather form much like normal gas giant planets 
in a gas disk beyond the ice line. They then migrate to their final close orbit sometime later in their lifetime. 
In order for the Hot Jupiter to move into a close orbit after it is formed, some sort of interaction must occur to change the planetary orbit. 
Disk migration theory postulates that the accretion of gases and solids during the planet's formation also alters the orbital parameters of the planet,
pushing the planet inward towards the star and thus being a potential mechanism for the creation of Hot Jupiters. However, within a star cluster, 
there are also potential sources for Hot Jupiter forming interactions via stellar encounters where one star approaches another star system. In this
case, outer orbit planets could be perturbed into a new orbit closer to the star. 
A side effect of such a planet-planet interaction is that another, "pair" planet, could be created that is 
thrown into a highly eccentric orbit by the same interaction that pushes the Hot Jupiter near the star. This would be a long-term effect on the host system that, if present, 
could be observed in the future to provide strong evidence in support of this theory. 

We will focus on these planet-planet interactions in order to see what effects that 
they may have on the planetary systems involved. Using computer simulations, we
may attempt to simulate the gravitational interactions involved with these planetary
systems in an attempt to replicate the processes present in the universe. We will
be able to generate data from a considerable number of systems and analyze it
in order to find probable Hot Jupiter formation scenarios. We may then
analyze the evolution of those systems that yielded a Hot Jupiter to find
what lasting relics of the formation event remain visible in the system.
By analyzing a large number of systems, there should be commonalities between
all or most of them that can be extracted. These commonalities provide a target
for observations in the future. If features similar to those generated by the
simulations are found, it will prove validity of the simulation in the modeling of Hot Jupiter
formation and potentially provide future refinements to be made in order to better understand
the process. If nothing similar
to the features generated by the simulation show up in observation of Hot Jupiter containing
systems, it may mean the computational model is invalid or too simple, and may
weaken the support for the planet-planet interaction theory of Hot Jupiter formation.

An N-Body simulation of star clusters populated with solar systems would be very
computationally expensive to run. In addition, with the exception of close
stellar encounters, the gravitational effect of planets is negligible on other
systems in comparison to the gravitational effect of the star. This means that
a huge amount of simulation would need to be performed that has virtually no
effect on the final outcome. As a result, we plan to use a two-stage approach to
the problem.

In the first phase, many N-body simulations of entire star clusters will be run. 
The stars populating these clusters will be naked, and no planets will come into
play during this phase.  Each simulation will be run until some stopping condition 
related to the dynamics of the
cluster is met. During these simulations, whenever two systems come close enough
to each other to cause a significant interaction between them, the relevant
parameters of each of the two systems will be recorded. These collision parameters
will be gathered from every simulation run and combine in order to generate
a database of parameter distributions from which random samples can be taken during
the next stage of the research. As the simulations of this phase do not 
account for planetary interaction, this database will be used to seed the second phase.

The second phase of the research involves Monte-Carlo simulations of
two-solar system universes. In each simulation, two solar systems will
be constructed, each containing three planets: an inner Earth-like planet, an
intermediate orbit Jupiter-like planet and an outer orbit Neptune-like planet.
For simplicity, all systems created for this stage will be more or less
exact copies of one another. The initial conditions for each simulation 
will be pulled from the database of parameter distributions created in the 
last stage. This should provide a large variety of realistic collision
scenarios to simulate, and should exclude situations that may lead to Hot Jupiter
formation but that are not physically attainable within a star cluster. 
After the initial conditions are used to 
populate the universe, the motion and interaction of all bodies will be
resolved until the systems separate from one another and stabilize 
into their final structure. The final state from each simulation will be
recorded in detail to allow further analysis over all results in order
to identify commonalities. The Monte-Carlo phase will ideally execute
many simulations with many different initial conditions 
in an attempt to provide good coverage of the entire initial
parameter space produced during the first stage.

Finally, after all simulations are run and data is collected, 
the full result set will be analyzed. The first step in this
process is to identify and group all Hot Jupiter formation scenarios.
The set of systems that lead to Hot Jupiter formation can then be
further analyzed to identify any and all lasting effects observed
in the planetary systems after they restabilized from the collision.
There will potentially be common features shared by many of these
systems, so we will need to devise a method that identifies which
simulations produced similar results and compute statistics about
them. In particular, we would like to identify what kind of initial 
condition may frequently lead to the result in question as well as
what might best be used to identify a system in the real universe
that has undergone a Hot Jupiter formation process similar to the
ones probed by the simulation.

% and the relevant parameters of all collision events will be recorded.
%Afterward, this collision data will be used to construct a distribution of
%possible parameters for stellar encounters, and the resulting distribution will be used to create a Monte-Carlo simulation of 
%manpiy collision events. Our objective is to see if some of the simulated collisions will result in Hot Jupiter formation scenarios. 
%If so,the outcomes of such simulations will then be analyzed to show what sort of influence the formation of the Hot Jupiters has on other planets in the system.

\begin{thebibliography}{9}

\bibitem{James B. Pollack et al. 1996}
James B. Pollack, Olenka Hubickyj, Peter Bodenheimer, Jack J. Lissauer, Morris Podolak, Yuval Greenzweig, Formation of the Giant Planets by Concurrent Accretion of Solids and Gas, Icarus, Volume 124, Issue 1, 1996, Pages 62-85, ISSN 0019-1035, http://dx.doi.org/10.1006/icar.1996.0190.
(http://www.sciencedirect.com/science/article/pii/S0019103596901906)

\bibitem{A. Brucalassi, 2016}
A.  Brucalassi, L.  Pasquini, R.  Saglia, M. T.  Ruiz, P.  Bonifacio, I.  Leão, B. L.  Canto Martins, J. R.  de Medeiros, L. R.  Bedin, K.  Biazzo, C.  Melo, C.  Lovis, S.  Randich,
Search for giant planets in M67 - III. Excess of hot Jupiters in dense open clusters
A\&A 592 L1 (2016)
DOI: 10.1051/0004-6361/201527561

\bibitem{Foreman-Mackey et al. 2016}
Foreman-Mackey, D., Morton, T.~D., Hogg, D.~W., Agol, E., \& Sch{\"o}lkopf, B.\ 2016, arXiv:1607.08237 

\bibitem{Shara et al. 2014} Shara, M.~M., Hurley, J.~R., \& Mardling, R.~A.\ 2014, arXiv:1411.7061 

\end{thebibliography}

\end{document}
