\documentclass[12pt]{article}
\usepackage{amsmath}
\usepackage{amssymb}
\usepackage[margin=1in]{geometry}
\usepackage{setspace}
\doublespacing

\author{Tyler Reisinger}
\title{Draft Conclusion}
\date{}

\begin{document}
\maketitle

We explored a potential formation scenario for the formation of Hot Jupiter
planets. These planets have been observed in the universe with some frequency,
but their exact method of formation remains debated. 
We theorize that strong interactions between two planetary systems in a star
cluster could perturb the orbit of a normal gas giant to approach very close to
its host star. 

Our simulations show that this is indeed a possibility, and that very close
interactions on the order of a few AU can frequently throw Jupiter-mass planets into
highly eccentric orbits with very close approaches to the host star. 
This may then also drastically impact the orbits of other planets
in the system. This provides support for the idea that a Hot Jupiter may very
often have another pair planet that was thrown into an extreme orbit by the same
process that affected the Jupiter's orbit. Thus,
observation of these pair planets may provide reason to search for a Hot Jupiter
nearer the star.
Further long-term simulation would be needed to see if the highly eccentric Jupiter
orbit could decay into a more circular orbit similar to many observed Hot Jupiters.
\end{document}

